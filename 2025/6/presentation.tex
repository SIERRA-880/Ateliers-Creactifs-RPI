%----------------------------------------------------------------------------------------
%    PACKAGES AND THEMES
%----------------------------------------------------------------------------------------

\documentclass[aspectratio=169,xcolor=dvipsnames]{beamer}
\usetheme{SimpleDarkBlue}

\usepackage{hyperref}
\usepackage{graphicx} % Allows including images
\usepackage{booktabs} % Allows the use of \toprule, \midrule and \bottomrule in tables

\usepackage{caption} % Allows the usage of \captionsetup
\DeclareCaptionFormat{nolabel}{#3} % Removes "Listing:" prefix
\captionsetup[lstlisting]{format=nolabel}

\usepackage{enumitem}
\setlist[itemize,1]{label=\textbullet}
\setlist[itemize,2]{label=--}

\usepackage[T1]{fontenc}
\usepackage{cascadia-code}

\usepackage{xcolor}
\definecolor{codegreen}{rgb}{0,0.6,0}
\definecolor{codegray}{rgb}{0.5,0.5,0.5}
\definecolor{codepurple}{rgb}{0.58,0,0.82}
\definecolor{backcolour}{rgb}{0.9,0.9,0.9}

\usepackage{listings}
\lstdefinestyle{mystyle}{
    backgroundcolor=\color{backcolour},
    commentstyle=\color{codegreen},
    keywordstyle=\color{magenta},
    numberstyle=\tiny\color{codegray},
    stringstyle=\color{codepurple},
    basicstyle=\ttfamily\footnotesize,
    breakatwhitespace=false,
    breaklines=true,
    captionpos=b,
    keepspaces=true,
    numbers=left,
    numbersep=5pt,
    showspaces=true,
    showstringspaces=true,
    showtabs=true,
    tabsize=4,
    aboveskip=0pt, % Reduce space above
    belowskip=0pt  % Reduce space below
}
\lstset{style=mystyle}

%----------------------------------------------------------------------------------------
%    TITLE PAGE
%----------------------------------------------------------------------------------------

\title{Ateliers Créactifs Raspberry Pi}
\subtitle{Création d'un système domotique local et sécurisé avec Home Assistant.}

\author{Jean Bourgies, François Marelli, Ugo Proietti}

\date{31 mars 2025}

%----------------------------------------------------------------------------------------
%    PRESENTATION SLIDES
%----------------------------------------------------------------------------------------

\begin{document}

\begin{frame}
    % Print the title page as the first slide
    \titlepage
\end{frame}

\begin{frame}{Table des matières}
    % Throughout your presentation, if you choose to use \section{} and \subsection{} commands, these will automatically be printed on this slide as an overview of your presentation
    \tableofcontents
\end{frame}

%------------------------------------------------
\section{Les solutions grand public}
%------------------------------------------------

\begin{frame}{Solutions grand public}
    \begin{columns}[c] % 'c' ensures vertical centering for both columns

    \column{0.5\textwidth}
        \begin{figure}
            \includegraphics[width=0.9\textwidth]{images/amazon_echo.jpg}
            \captionsetup{labelformat=empty}
            \caption{Amazon Echo}
        \end{figure}

    \column{0.5\textwidth}
        \begin{figure}
            \includegraphics[width=0.9\textwidth]{images/google_home.jpg}
            \captionsetup{labelformat=empty}
            \caption{Google Home}
        \end{figure}

    \end{columns}
\end{frame}

\begin{frame}{Le problème}
    \begin{columns}[c] % 'c' ensures vertical centering for both columns

    \column{1\textwidth}
        \begin{itemize}
            \item Modèle close source
            \item Collecte et vente des données personnelles
            \item Dépendance à un service cloud
            \item Manque de compatibilité
        \end{itemize}

    \end{columns}
\end{frame}


%------------------------------------------------
\section{Les alternatives}
%------------------------------------------------

\begin{frame}{Solutions alternatives}
    \begin{columns}[c] % 'c' ensures vertical centering for both columns

    \column{1\textwidth}
        \begin{itemize}
            \item Home Assistant, OpenHAB, Domoticz sont trois solutions open source
            \item Home Assistant est la solution la plus populaire
        \end{itemize}
        \vspace{1cm}
        \begin{itemize}
            \item Installation locale, par exemple sur un Raspberry Pi
            \item Possibilité de travailler sans connexion internet
            \item Compatibilité avec de nombreux protocoles
            \item Add-ons pour étendre les fonctionnalités
        \end{itemize}

    \end{columns}
\end{frame}

%------------------------------------------------
\section{Aspects techniques}
%------------------------------------------------

\begin{frame}{Protocoles répandus}
    \begin{columns}[c] % 'c' ensures vertical centering for both columns

    \column{0.5\textwidth}
        \begin{itemize}
            \item \textit{Méthode de communication entre les objets connectés}
            \item Wifi
            \item Zigbee
            \item Z-Wave
            \item Thread
            \item Bluetooth
        \end{itemize}

    \column{0.5\textwidth}
        \begin{itemize}
            \item \textit{Protocole de communication entre les objets connectés}
            \item MQTT
            \item Zigbee
            \item Z-Wave
            \item Matter
        \end{itemize}

    \end{columns}
\end{frame}

%------------------------------------------------
\section{Home Assistant}
%------------------------------------------------

\begin{frame}{Home Assistant}
    \begin{columns}[c] % 'c' ensures vertical centering for both columns

    \column{1\textwidth}
        \begin{figure}
            \includegraphics[width=1\textwidth]{images/home_assistant.png}
            \captionsetup{labelformat=empty}
            \caption{Home Assistant}
        \end{figure}
        Open source home automation that puts local control and privacy first. Powered by a worldwide community of tinkerers and DIY enthusiasts. \\
        \vspace{0.3cm}
        \textit{La domotique open source qui privilégie le contrôle local et le respect de la vie privée. Alimenté par une communauté mondiale de bricoleurs et de passionnés.}

    \end{columns}
\end{frame}

\begin{frame}{Home Assistant}
    \begin{columns}[c] % 'c' ensures vertical centering for both columns

    \column{1\textwidth}
        \begin{itemize}
            \item \url{https://www.openhomefoundation.org/}
            \item \url{https://www.home-assistant.io/}
            \item \url{https://demo.home-assistant.io/}
        \end{itemize}

    \end{columns}
\end{frame}

%------------------------------------------------
\section{ESPHome et Tasmota}
%------------------------------------------------

\begin{frame}{ESPHome}
    \begin{columns}[c] % 'c' ensures vertical centering for both columns

    \column{1\textwidth}
        \begin{itemize}
            \item ESPHome est un système qui permet de transformer des microcontrôleurs courants en appareils domestiques intelligents.
            \item \url{https://esphome.io/}
        \end{itemize}

    \end{columns}
\end{frame}

\begin{frame}{Tasmota}
    \begin{columns}[c] % 'c' ensures vertical centering for both columns

    \column{1\textwidth}
        \begin{itemize}
            \item Logiciel libre pour les appareils ESP. Contrôle local total avec installation et mises à jour rapides. Contrôle par MQTT, interface Web, HTTP ou série. Automatisation à l'aide de minuteries, de règles ou de scripts. Intégration avec des solutions domotiques. Extrêmement extensible et flexible.
            \item \url{https://tasmota.github.io/}
        \end{itemize}

    \end{columns}
\end{frame}


\end{document}
