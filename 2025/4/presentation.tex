%----------------------------------------------------------------------------------------
%    PACKAGES AND THEMES
%----------------------------------------------------------------------------------------

\documentclass[aspectratio=169,xcolor=dvipsnames]{beamer}
\usetheme{SimpleDarkBlue}

\usepackage{hyperref}
\usepackage{graphicx} % Allows including images
\usepackage{booktabs} % Allows the use of \toprule, \midrule and \bottomrule in tables

\usepackage{caption} % Allows the usage of \captionsetup
\DeclareCaptionFormat{nolabel}{#3} % Removes "Listing:" prefix
\captionsetup[lstlisting]{format=nolabel}

\usepackage{enumitem}
\setlist[itemize,1]{label=\textbullet}
\setlist[itemize,2]{label=--}

\usepackage[T1]{fontenc}
\usepackage{cascadia-code}

\usepackage{xcolor}
\definecolor{codegreen}{rgb}{0,0.6,0}
\definecolor{codegray}{rgb}{0.5,0.5,0.5}
\definecolor{codepurple}{rgb}{0.58,0,0.82}
\definecolor{backcolour}{rgb}{0.9,0.9,0.9}

\usepackage{listings}
\lstdefinestyle{mystyle}{
    backgroundcolor=\color{backcolour},
    commentstyle=\color{codegreen},
    keywordstyle=\color{magenta},
    numberstyle=\tiny\color{codegray},
    stringstyle=\color{codepurple},
    basicstyle=\ttfamily\footnotesize,
    breakatwhitespace=false,
    breaklines=true,
    captionpos=b,
    keepspaces=true,
    numbers=left,
    numbersep=5pt,
    showspaces=true,
    showstringspaces=true,
    showtabs=true,
    tabsize=4,
    aboveskip=0pt, % Reduce space above
    belowskip=0pt  % Reduce space below
}
\lstset{style=mystyle}

%----------------------------------------------------------------------------------------
%    TITLE PAGE
%----------------------------------------------------------------------------------------

\title{Ateliers Créactifs Raspberry Pi}
\subtitle{Programmation 1 : utilisation de capteurs et acquisition de données.}

\author{Jean Bourgies, François Marelli, Ugo Proietti}

\date{17 mars 2025}

%----------------------------------------------------------------------------------------
%    PRESENTATION SLIDES
%----------------------------------------------------------------------------------------

\begin{document}

\begin{frame}
    % Print the title page as the first slide
    \titlepage
\end{frame}

\begin{frame}{Table des matières}
    % Throughout your presentation, if you choose to use \section{} and \subsection{} commands, these will automatically be printed on this slide as an overview of your presentation
    \tableofcontents
\end{frame}


%------------------------------------------------
\section{Premier programme en Python}
%------------------------------------------------

\begin{frame}{Python3}
    \begin{columns}[c] % 'c' ensures vertical centering for both columns

        \column{1\textwidth}
        \begin{itemize}
            \item Language de programmation.
            \item Très utilisé pour sa facilité d'apprentissage et sa simplicité.
        \end{itemize}

    \end{columns}
\end{frame}

\begin{frame}{Commandes de base}
    \begin{columns}[c] % 'c' ensures vertical centering for both columns

        \column{1\textwidth}
        \begin{itemize}
            \item \texttt{python3} : interpréteur Python.
            \item \texttt{pip3} : gestionnaire de paquets Python.
            \item \texttt{python3 -m venv venv} : environnement virtuel pour le développement.
        \end{itemize}

    \end{columns}
\end{frame}


%------------------------------------------------
\section{Explication des ports GPIO}
%------------------------------------------------

\begin{frame}{General Purpose Input/Output}
    \begin{columns}[c] % 'c' ensures vertical centering for both columns

        \column{1\textwidth}
        Broches situées sur le Raspberry Pi qui permettent de communiquer avec d'autres composants électroniques comme des capteurs, des LED ou des moteurs.\\
        \vspace{5mm}
        Les ports GPIO permettent de :
        \begin{itemize}
            \item Envoyer un signal électrique pour allumer une LED ou activer un moteur
            \item Recevoir un signal d'un bouton-poussoir ou d'un capteur (comme le DHT22 pour la température)
            \item Communiquer avec d'autres composants via des protocoles comme I2C ou SPI
        \end{itemize}
    \end{columns}
\end{frame}

\begin{frame}{Carte des ports GPIO}
    \begin{columns}[c] % 'c' ensures vertical centering for both columns

        \column{1\textwidth}
        Plusieurs façon de voir les ports GPIO :
        \vspace{5mm}
        \begin{itemize}
            \item Sur le site \url{https://pinout.xyz/}
            \item En utilisant la commande \texttt{pinout}
        \end{itemize}
    \end{columns}
\end{frame}


%------------------------------------------------
\section{Présentation du capteur DHT22}
%------------------------------------------------
%------------------------------------------------
\section{Acquisition de données}
%------------------------------------------------

\begin{frame}{Installation des dépendances}
    \begin{columns}[c] % 'c' ensures vertical centering for both columns

        \column{1\textwidth}
        \begin{itemize}
            \item \texttt{python3 -m venv venv}
            \item \texttt{source venv/bin/activate}
            \item \texttt{sudo apt update}
            \item \texttt{sudo apt install libgpiod2}
            \item \texttt{pip3 install -{}-upgrade pip setuptools wheel}
            \item \texttt{pip3 install adafruit-circuitpython-dht}
        \end{itemize}

    \end{columns}
\end{frame}

\begin{frame}[fragile]
\frametitle{Écriture du programme}
    \begin{columns}[c] % 'c' ensures vertical centering for both columns

        \column{0.65\textwidth} % Left column
        \begin{lstlisting}[language=Python, caption=DHT22.py]
import time
import adafruit_dht
import board

dht_device = adafruit_dht.DHT22(board.D4)

while True:
    try:
        temperature = dht_device.temperature
        humidity = dht_device.humidity
        print(f"Temp: {temperature:.1f}C / Humi: {humidity:.1f}%")
    except RuntimeError as err:
        print(err.args[0])

    time.sleep(2.0)
        \end{lstlisting}

        \column{0.35\textwidth}
        Pour ceux qui ne veulent pas recopier :
        \url{https://chk.me/dJGDS4S} \\
        \vspace{5mm}
        Pour éxecuter le programme :
        \texttt{python3 DHT22.py}
    \end{columns}
\end{frame}

\end{document}
