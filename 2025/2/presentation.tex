%----------------------------------------------------------------------------------------
%    PACKAGES AND THEMES
%----------------------------------------------------------------------------------------

\documentclass[aspectratio=169,xcolor=dvipsnames]{beamer}
\usetheme{SimpleDarkBlue}

\usepackage{hyperref}
\usepackage{graphicx} % Allows including images
\usepackage{booktabs} % Allows the use of \toprule, \midrule and \bottomrule in tables

\usepackage{caption} % Allows the usage of \captionsetup
\DeclareCaptionFormat{nolabel}{#3} % Removes "Listing:" prefix
\captionsetup[lstlisting]{format=nolabel}

\usepackage{enumitem}
\setlist[itemize,1]{label=\textbullet}
\setlist[itemize,2]{label=--}

\usepackage[T1]{fontenc}
\usepackage{cascadia-code}

\usepackage{xcolor}
\definecolor{codegreen}{rgb}{0,0.6,0}
\definecolor{codegray}{rgb}{0.5,0.5,0.5}
\definecolor{codepurple}{rgb}{0.58,0,0.82}
\definecolor{backcolour}{rgb}{0.9,0.9,0.9}

\usepackage{listings}
\lstdefinestyle{mystyle}{
    backgroundcolor=\color{backcolour},
    commentstyle=\color{codegreen},
    keywordstyle=\color{magenta},
    numberstyle=\tiny\color{codegray},
    stringstyle=\color{codepurple},
    basicstyle=\ttfamily\footnotesize,
    breakatwhitespace=false,
    breaklines=true,
    captionpos=b,
    keepspaces=true,
    numbers=left,
    numbersep=5pt,
    showspaces=true,
    showstringspaces=true,
    showtabs=true,
    tabsize=4,
    aboveskip=0pt, % Reduce space above
    belowskip=0pt  % Reduce space below
}
\lstset{style=mystyle}

%----------------------------------------------------------------------------------------
%    TITLE PAGE
%----------------------------------------------------------------------------------------

\title{Ateliers Créactifs Raspberry Pi}
\subtitle{Intégration d'une caméra pour transformer son Raspberry PI en photomaton ou en système de videósurveillance.}

\author{Jean Bourgies, François Marelli, Ugo Proietti}

\date{17 février 2025}

%----------------------------------------------------------------------------------------
%    PRESENTATION SLIDES
%----------------------------------------------------------------------------------------

\begin{document}

\begin{frame}
    % Print the title page as the first slide
    \titlepage
\end{frame}

\begin{frame}{Table des matières}
    % Throughout your presentation, if you choose to use \section{} and \subsection{} commands, these will automatically be printed on this slide as an overview of your presentation
    \tableofcontents
\end{frame}

%------------------------------------------------
\section{Port CSI}
%------------------------------------------------

\begin{frame}{Port CSI}
    \begin{columns}[c] % 'c' ensures vertical centering for both columns

        \column{.6\textwidth} % Left column
        \begin{figure}
            \includegraphics[width=1\textwidth]{images/rpi-5-dsi-csi.jpg}
        \end{figure}

        \column{.4\textwidth} % Right column
        \begin{itemize}
            \item Camera Serial Interface
            \item Modification depuis le Raspberry Pi 5
            \item Sur les anciens modèles, chercher l'indications "CAMERA"
        \end{itemize}

    \end{columns}
\end{frame}

\begin{frame}{Caméra CSI}
    \begin{columns}[c] % 'c' ensures vertical centering for both columns

        \column{.6\textwidth} % Left column
        \begin{figure}
            \includegraphics[width=0.6\linewidth]{images/camera-csi.png}
        \end{figure}

        \column{.4\textwidth} % Right column
        \begin{itemize}
            \item 30€-90€
            \item Compacte
            \item Plusieurs modèles et objectifs
            \item Léger pour le processeur
        \end{itemize}

    \end{columns}
\end{frame}


%------------------------------------------------
\section{Prise de photo}
%------------------------------------------------

\begin{frame}{Prise de photo et vidéo}
    \begin{columns}[c] % 'c' ensures vertical centering for both columns

        \column{1\textwidth}
        \begin{itemize}
            \item \texttt{libcamera-jpeg -o image.jpg} : prend une photo et l'enregistre sous le nom \texttt{image.jpg}
            \item \texttt{libcamera-vid -t 5000 -o video.h264} : prend une vidéo de 5000 milisecondes et l'enregistre sous le nom \texttt{video.h264}
        \end{itemize}

    \end{columns}
\end{frame}

\begin{frame}{Prise de photo et vidéo (avancé)}
    \begin{columns}[c] % 'c' ensures vertical centering for both columns

        \column{1\textwidth}
        \begin{itemize}
            \item \texttt{libcamera-jpeg}
            \begin{itemize}
                \item \texttt{-t 5000} : minuteur de 5 secondes
                \item \texttt{-rot 180} : rotation de 180 degrés
                \item \texttt{-o image.jpeg} : nom du fichier de sortie
                \item \texttt{-awb greyworld} : à utiliser si l'image est rose
            \end{itemize}
            \item \texttt{libcamera-vid}
            \begin{itemize}
                \item \texttt{-t 3000} : temps de 3 secondes
                \item \texttt{-rot 90} : rotation de 90 degrés
                \item \texttt{-o video.h264} : nom du fichier de sortie
                \item \texttt{-awb greyworld} : à utiliser si l'image est rose
            \end{itemize}
            \item \texttt{man libcamera-jpeg} et \texttt{man libcamera-vid} pour la liste d'options
        \end{itemize}

    \end{columns}
\end{frame}


%------------------------------------------------
\section{Port DSI}
%------------------------------------------------

\begin{frame}{Port DSI}
    \begin{columns}[c] % 'c' ensures vertical centering for both columns

        \column{.6\textwidth} % Left column
        \begin{figure}
            \includegraphics[width=1\textwidth]{images/rpi-5-dsi-csi.jpg}
        \end{figure}

        \column{.4\textwidth} % Right column
        \begin{itemize}
            \item Display Serial Interface
            \item Modification depuis le Raspberry Pi 5
            \item Sur les anciens modèles, chercher l'indications "DISPLAY"
        \end{itemize}

    \end{columns}
\end{frame}

\begin{frame}{Écran DSI}
    \begin{columns}[c] % 'c' ensures vertical centering for both columns

        \column{.6\textwidth} % Left column
        \begin{figure}
            \includegraphics[width=0.8\textwidth]{images/display-dsi.png}
        \end{figure}

        \column{.4\textwidth} % Right column
        \begin{itemize}
            \item 80€
            \item Support des fonctions tactiles
            \item Fixation prévue pour le Raspberry Pi
            \item Faible consommation énergétique
            \item \texttt{sudo apt install matchbox-keyboard} : clavier tactile
        \end{itemize}

    \end{columns}
\end{frame}

%------------------------------------------------
\section{Carousel d'images}
%------------------------------------------------

\begin{frame}[fragile]
\frametitle{Carousel d'images}
    \begin{columns}[c] % 'c' ensures vertical centering for both columns

        \column{.4\textwidth} % Left column
        \begin{itemize}
            \item \texttt{sudo apt install feh}
            \item \texttt{touch carousel.sh}
            \item \texttt{nano carousel.sh}
            \item \texttt{chmod +x carousel.sh}
            \item \texttt{./carousel.sh}
        \end{itemize}

        \column{.6\textwidth} % Right column
        \begin{lstlisting}[language=Bash, caption=carousel.sh]
#!/bin/bash

images=($(shuf -e *.jpeg))

for image in "${images[@]}"; do
    feh --fullscreen --zoom fill "$image" &
    sleep 3
    killall feh
done
        \end{lstlisting}

    \end{columns}
\end{frame}

%------------------------------------------------
\section{Photomaton}
%------------------------------------------------

\begin{frame}{Photomaton}
    \begin{columns}[c] % 'c' ensures vertical centering for both columns

        \column{1\textwidth} % Right column
        \begin{itemize}
            \item Raspberry Pi
            \item Écran DSI
            \item Caméra CSI
            \item Programme en Python pour utiliser l'écran tactile
            \item Imprimante pour imprimer en plus d'envoyer par mail
            \item Donner ce contexte à une IA et demander de donner le programme
        \end{itemize}

    \end{columns}
\end{frame}

%------------------------------------------------
\section{Vidéosurveillance}
%------------------------------------------------

\begin{frame}{Projets de NVR}
    \begin{columns}[c] % 'c' ensures vertical centering for both columns

        \column{.35\textwidth} % Left column
        NVR : Network Video Recorder

        \column{.65\textwidth} % Right column
        \begin{itemize}
            \item MotionEye
            \item https://github.com/motioneye-project/motioneye
            \item Compatible avec du matériel léger.
            \item Possibilité de fabriquer ses propre caméras.
        \end{itemize}
        \vspace{10pt}
        \begin{itemize}
            \item Frigate
            \item https://github.com/blakeblackshear/frigate
            \item Besoin de matériel capable de faire de la reconnaissance d'image (RPI 5 + Coral TPU)
            \item Se connecte à des caméras en réseau.
        \end{itemize}

    \end{columns}
\end{frame}

%------------------------------------------------
\section{Reconnaissance faciale}
%------------------------------------------------

\begin{frame}{Reconnaissance faciale}
    \begin{columns}[c] % 'c' ensures vertical centering for both columns

        \column{1\textwidth} % Left column
        Demonstration d'un programme de reconnaissance faciale en Python sur un RaspberryPi4. \\
        Le code se trouve sur le Github de ce cours.

    \end{columns}
\end{frame}

\end{document}
